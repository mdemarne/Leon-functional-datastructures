The implementation of Binomial Heaps we used, 
along with the one of Trees, 
can be found in pages 68-72 of the reference book (\cite{Okasaki}).

\subsection{Data Structure Overview}
A Binomial Heap is an implementation of mergeable (or meldable) priority queue.
It uses Trees, so we had to implement them as well.
Let us see first their implementation.

\subsubsection{Trees Overview}
A Tree is simply represented as a Node with an element 
(the priority, of totally ordered type), a rank and a list of Trees as children.
The rank, which is $\geq 0$, is to be understood as follows:
a Node of rank $k$ has $k$ children of ranks $k$, $k-1$, ..., $0$,
a rank $0$ meaning a Node without children.
From that it comes that a Tree with a root Node of rank $k$ 
contains in total $2^k$ elements,
so a Tree can represent a power of two, 
like a bit in a binary number.

The Trees must satisfy the Minimum Heap Property, 
which means that the element of a parent Node 
is less or equal than any of its children Node's elements.
Also, in the implementation, the children list is maintained in decreasing order of rank.
These properties are thus two invariants to check on Trees,
along with the domain of the rank.

The \verb|link| operation on Trees takes two Trees of rank $k$, 
makes the one with bigger root element the first child of the other,
so that both invariants are satisfied,
and then returns the Tree, of rank $k+1$ now.
It is thus like the addition of two bits of a binary number.
The operation is done in constant time.

Let us see now how the Trees are used in the Binomial Heap implementation.

\subsubsection{Binomial Heaps Overview}
A Binomial Heap is implemented as a list of Trees, 
which is kept in increasing order of rank in the implementation.
Also, there must not be more than one Tree of a particular rank.
These properties are thus invariants to check.
 
We can then visualise a well-formed Binomial Heap as the binary number of its size, 
with the Trees in the list as the 1-valued bits of the number.

With help of the \verb|link| operation on Trees, 
we can see a \verb|merge| of two Binomial Heaps 
as an addition of the two binary numbers that represent them.
In this view, \verb|link| is used to add two bits of same position,
and the result of the operation is the carry.

The other operations on Binomial Heaps are \verb|insert|, \verb|findMin| and \verb|deleteMin|. 
The \verb|insert| operation creates a 0-ranked Tree with the element to insert 
and puts the Tree in the Heap, using an \verb|insertTree| function, 
also used by \verb|merge| to insert the carry in the recursively merged Heaps,
at each carry step.
This function inserts the Tree only if its rank is $\leq$ to the smallest rank in the Tree list of the Heap 
(either it adds it to the list, either it recursively \verb|link|s and \verb|insertTree|s),
so it had to be checked that the function is not called with a Tree of bigger rank in the implementation.

The \verb|findMin| operation needs only to look for the minimum element 
in the root Nodes of the Trees in the Tree list.
This is because the Trees satisfy the Minimum Heap Property.

Finally, the \verb|deleteMin| operation finds the Tree with the minimum root in the list,
removes its root 
and reverses its children so that the children list is of the correct form for a Tree list of a Binomial Heap.
After that, it \verb|merge|s this new Heap with the Heap composed of 
the remaining Trees of the original Binomial Heap.

All operation are done in amortized logarithmic time.
Let us see now the verification of the implementations 
for both Trees and Binomial Heaps with Leon.

\subsection{Verification with Leon}
As for the first data structure verification presented in this report,
 \verb|size|, \verb|content| and \verb|toList| functions were used
 for pre- and post-conditions of operations,
as well as the invariants for Trees and Binomial Heaps we saw above.

Some tests were written for the Binomial Heaps and 
it helped to find some mistakes we made in writing the code and 
which were not found by Leon.
For example, an invariant was written too strong and 
the error could be seen only with tests,
when a precondition for an operation was not satisfied.

We had some difficulties to state the total order of the elements' type
(T $<:$ Ordered[T] was not understood by Leon), 
so we decided to use BigInts to be able to verify the data structure.
Also, we saw a limitation from Leon preventing us to have a nicer code, 
with implicit classes, for example, even if probably tricky to implement in Leon.

Leon could unfortunately not state as valid most of the operations, 
which are still \verb|Unknown|s.

%TODO
%more ?
