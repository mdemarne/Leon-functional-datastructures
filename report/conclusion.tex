We were able to implement two data structures in our project,
namely Catenable Lists and Binomial Heaps,
and to check some operations on them.
Unfortunately Leon failed to prove or disprove some properties. 
However our experiments have shown 
that this is difficult to avoid with recursive definitions.
The results we described in this report were obtained with the \texttt{--assumepre} option,
which allowed to prove a few more conditions.

We were able to isolate some limitations and find ways to circumvent them, 
being either library ones (like operations on \texttt{Set}s which were not supported) for compilation with \texttt{scalac}
or syntactic ones.

It was nevertheless very interesting to use Leon and it helped us find some errors 
in translating the book's implementation into Scala (\citep{Okasaki}). 

Our implementation of the two data structures we have studied is of course purely functional.